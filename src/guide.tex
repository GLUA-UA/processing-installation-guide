\documentclass{article}

\usepackage{multirow}
\usepackage[utf8]{inputenc}
\usepackage{graphicx} % Required for the inclusion of images
\usepackage{natbib} % Required to change bibliography style to APA
\usepackage{mathtools}
\usepackage{listings}
\usepackage{color}
\usepackage[margin=1in]{geometry}
\usepackage[table,xcdraw]{xcolor}
\usepackage{multicol}
\usepackage{hyperref}
\usepackage{wrapfig}
\usepackage{capt-of}
\usepackage{caption}
\usepackage{subcaption}
\usepackage{wrapfig}
\usepackage{float}

\setlength\parindent{0pt} % Removes all indentation from paragraphs

\definecolor{dkgreen}{rgb}{0,0.6,0}
\definecolor{gray}{rgb}{0.5,0.5,0.5}
\definecolor{mauve}{rgb}{0.58,0,0.82}

\graphicspath{ {images/} }

\lstset{frame=tb,
  aboveskip=3mm,
  belowskip=3mm,
  showstringspaces=false,
  columns=flexible,
  basicstyle={\small\ttfamily},
  numbers=none,
  numberstyle=\tiny\color{gray},
  keywordstyle=\color{blue},
  commentstyle=\color{dkgreen},
  stringstyle=\color{mauve},
  breaklines=true,
  breakatwhitespace=true,
  tabsize=3
}

\title{Processing installation guide for Linux}

\author{Grupo de Linux da Universidade de Aveiro\\Eduardo Sousa - eduardosousa@ua.pt}

\date{2016}

\begin{document}

\maketitle

\section{Install Oracle JDK 8}

\textbf{Note:} This step is only required if you don't have the Oracle JDK installed.

Open a terminal and run the following commands:

\begin{lstlisting}
	sudo add-apt-repository ppa:webupd8team/java
	sudo apt-get update
	sudo apt-get install oracle-java8-installer
\end{lstlisting}

To check if the Oracle JDK 8 was successfully installed, run this command:

\begin{lstlisting}
	java --version
\end{lstlisting}

\section{Install Processing}

\subsection{Download Processing}

Go to \url{http://processing.org/download/?processing} and download the latest version for Linux.

\subsection{Installing Processing}

Using the terminal, go to the your downloads folder.

\begin{lstlisting}
	cd ~/Downloads
\end{lstlisting}

Uncompress the file and delete the compressed file, by running these commands in the terminal:

\begin{lstlisting}
	tar xvfz processing-*.tgz
	rm processing-*.tgz
\end{lstlisting}

Now let's move it to /opt, by running this command in the terminal:

\begin{lstlisting}
	mv processing-* /opt/processing
\end{lstlisting}

Let's make a symbolic link, so that we don't need to specify the classpath, by doing this:

\begin{lstlisting}
	sudo ln -s /opt/processing/core/library/core.jar /usr/lib/jvm/java-8-oracle/jre/lib/ext/core.jar
\end{lstlisting}

Create an alias, so you can call it anywhere. Let's edit the .bashrc file.

\begin{lstlisting}
	nano ~/.bashrc
\end{lstlisting}

\clearpage

Add this line in the end of the file and save it.

\begin{lstlisting}
	alias processing="/opt/processing/processing"
\end{lstlisting}

Now, let's refresh our environment, by running this in the terminal:

\begin{lstlisting}
	source ~/.bashrc
\end{lstlisting}

To test if processing was successfully installed, run this in the terminal:

\begin{lstlisting}
	processing
\end{lstlisting}

\end{document}
